\documentclass{beamer}
\usetheme{Boadilla}

\title{Tangent Planes}
\author{Joey Bernard}
\institute{University of New Brunswick}
\date{\today}

\begin{document}

\begin{frame}
  \titlepage
\end{frame}

\begin{frame}
  \frametitle{Surfaces of the form z=f(x,y)}
  \begin{itemize}
  \item To find a tangent plane at a point, need two tangent vectors here
  \item The two tangent vectors are found with the partial derivatives:
    \begin{itemize}
    \item $[1, 0, f_x(x_0, y_0)]$
    \item $[0, 1, f_y(x_0, y_0)]$
    \end{itemize}
  \item Taking the cross product gives you the normal vector: $[-f_x(x_0, y_0), -f_y(x_0, y_0), 1]$
  \item The resulting equation of a plane is
    \begin{itemize}
    \item $z = f(x_0, y_0) + f_x(x_0, y_0)(x - x_0) + f_y(x_0, y_0)(y - y_0)$
    \end{itemize}
  \end{itemize}
\end{frame}

\begin{frame}
  \frametitle{Surfaces of the form G(x, y, z) = 0}
  \begin{itemize}
  \item Everything here also works for equations of the form $G(x, y, z) = K$, where K is a constant
  \item The gradient of G, $\nabla G(x_0, y_0, z_0)$ is the normal of the tangent plane at the point $(x_0, y_0, z_0)$
  \item The equation of this tangent plane is given by
    \begin{itemize}
    \item $\nabla G(x_0, y_0, z_0) \dot [x-x_0, y-y_0, z-z_0] = 0$
    \end{itemize}
  \end{itemize}
\end{frame}


\end{document}
