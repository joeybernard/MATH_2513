\documentclass{beamer}
\usetheme{Boadilla}

\title{Polar Integrals}

\author{Joey Bernard}
\institute{University of New Brunswick}
\date{\today}

\begin{document}

\begin{frame}
  \titlepage
\end{frame}

\begin{frame}
  \frametitle{Polar coordinates}
  \begin{itemize}
  \item The coordinates are r and $\theta$
  \item Converting to Cartesian and back is done by
    \begin{itemize}
    \item $x=r cos(\theta)       y=r sin(\theta)$
    \item $r=sqrt{x^2 + y^2}      \theta=arctan(\frac{y}{x})$
    \end{itemize}
  \end{itemize}
\end{frame}

\begin{frame}
  \frametitle{Integrals in Polar Coordinates}
  \begin{itemize}
  \item When doing area integrals we use the definition
    \begin{itemize}
    \item $dA = r dr d\theta$
    \end{itemize}
  \item as opposed to
    \begin{itemize}
    \item $dA = dx dy$
    \end{itemize}
    \item used in Cartesian coordinates
  \end{itemize}
\end{frame}

\end{document}

