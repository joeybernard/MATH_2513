\documentclass{beamer}
\usetheme{Boadilla}

\title{3D Sketching}
\author{Joey Bernard}
\institute{University of New Brunswick}
\date{\today}

\begin{document}

\begin{frame}
  \titlepage
\end{frame}

\begin{frame}
  \frametitle{Functions and graphs}
  \begin{itemize}
  \item A function f maps 2 values, (x,y), to a single value, z. D is the domain of (x,y) values, and R is the range of z values
    \begin{itemize}
    \item ${z = f(x,y) | (x,y) \in D}$
    \end{itemize}
  \item The set of all points (x,y,f(x,y)) in space, for (x,y) in the domain of f, is called the graph of f
  \item The graph of f is also called the surface z = f(x,y)
  \end{itemize}
\end{frame}

\begin{frame}
  \frametitle{Level curves}
  \begin{itemize}
  \item Level curves of a function f, of two variables, are the curves with equations f(x,y) = k, with k some constant
  \item These k are z values that allow for curves to drawn in the xy-plane
  \end{itemize}
\end{frame}

\end{document}

