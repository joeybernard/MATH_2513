\documentclass{beamer}
\usetheme{Boadilla}

\title{Lines}
\subtitle{Using Beamer}
\author{Joey Bernard}
\institute{University of New Brunswick}
\date{\today}

\begin{document}

\begin{frame}
  \titlepage
\end{frame}

\begin{frame}
  \frametitle{Definition of a line}
  \begin{itemize}
  \item Lines are defined as a set of points
  \item Lines can be defined in 2D or 3D
  \item To describe a line, you need to know a point and a direction
  \item Vectors are helpful to define a line
  \end{itemize}
\end{frame}

\begin{frame}
  \frametitle{Vector basics}
  \begin{itemize}
  \item Vectors are identified as bolded text, or a line or arrow over the variable
  \item Vectors can be defined by a triple of numbers, for each basis direction
  \item Vectors are added by connecting tails to heads
  \item A point can be represented by a vector going from the origin to that point
  \end{itemize}
\end{frame}

\begin{frame}
  \frametitle{Vector equation of a line}
  \begin{itemize}
  \item If we know one point and the direction, we can describe the whole line
  \item The vector equation for a line is given by:
  \item $\vec{r} = \vec{r_0} + t * \vec{d}$
  \end{itemize}
\end{frame}

\begin{frame}
  \frametitle{Parametric equation of a line}
  If we have the coordinates for each of the vectors, we can define:
  \begin{itemize}
  \item $\vec{r} = <x, y, z>$
  \item $\vec{d} = <a, b, c>$
  \item $\vec{r_0} = <x_0, y_0, z_0>$
  \end{itemize}
  This leads to the parametric equations:
  \begin{itemize}
  \item $x = x_0 + at$
  \item $y = y_0 + bt$
  \item $z = z_0 + ct$
  \end{itemize}
\end{frame}

\begin{frame}
  \frametitle{Symmetric equations of a line}
  We can solve for t, to get the symmetric equations:
  \begin{itemize}
  \item $t = \frac{x - x_0}{a} = \frac{y - y_0}{b} = \frac{z - z_0}{c}$
  \end{itemize}
\end{frame}

\begin{frame}
  \frametitle{Skew lines}
  Unlike in 2D, in 3D, lines can be not and yet still not intersect. These are called skew lines. In order to prove, you need to:
  \begin{itemize}
  \item show not parallel by showing that there is no C such that $\vec{d_1} = C\vec{d_2}$
  \item show no intersection by looking at the system of equations for both lines, assuming that there is some $<x, y, z>$ that is on both lines
  \end{itemize}
\end{frame}

\end{document}

