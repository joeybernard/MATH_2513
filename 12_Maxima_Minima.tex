\documentclass{beamer}
\usetheme{Boadilla}

\title{Maxima and Minima}

\author{Joey Bernard}
\institute{University of New Brunswick}
\date{\today}

\begin{document}

\begin{frame}
  \titlepage
\end{frame}

\begin{frame}
  \frametitle{Critical Points}
  \begin{itemize}
  \item In 2D, maxima and minima are when the tangents of the curve are flat
  \item This is when the first derivative (slope of the tangent line) is equal to zero
  \item In 3D, this is similar, except that we need to use the gradient
  \item So, maxima and minima happen when the gradient is zero, but this can happen outside of maxima and minima
  \item We introduce a new term, a critical point, defined by the gradient being equal to zero
  \end{itemize}
\end{frame}

\begin{frame}
  \frametitle{Second Derivative test}
  \begin{itemize}
  \item We can use the second derivative to help decide local maxima and minima in 2D
  \item If f'' < 0, we will have a local maxima
  \item If f'' > 0, we will have a local minima
  \item If f'' = 0, we don't know
  \item The equivalent in 3D is
    \begin{itemize}
    \item Use the descriminant of f: $D(x,y) = f_{xx}(x,y)f_{yy}(x,y) - f_{xy}^2$
    \item D > 0 and $f_{xx} > 0$ : f has a local minimum
    \item D > 0 and $f_{xx} < 0$ : f has a local maximum
    \item D < 0 : f has a saddle point
    \item D = 0 : we don't know
    \end{itemize}
  \end{itemize}
\end{frame}

\begin{frame}
  \frametitle{Global Maxima and Minima}
  \begin{itemize}
  \item 'Global' maxima and minima are always defined on an interval
  \item With two independent variables, may use the unit disc ($x^2 + y^2 \le 1$
  \item may need to use a parameterization to figure out the behaviour of the equation
  \end{itemize}
\end{frame}


\end{document}

