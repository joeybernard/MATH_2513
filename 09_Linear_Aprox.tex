\documentclass{beamer}
\usetheme{Boadilla}

\title{Linear Approximation}
\author{Joey Bernard}
\institute{University of New Brunswick}
\date{\today}

\begin{document}

\begin{frame}
  \titlepage
\end{frame}

\begin{frame}
  \frametitle{Linear Approximations}
  \begin{itemize}
  \item We want to approximate a function near some starting point $(x_0, y_0)$
  \item The simplest approximation is to assume that the function is constant
  \item The next simplest is that it is linear
  \item We get the linear form in 2D by looking at the tangent line at a point along a curve
  \end{itemize}
\end{frame}

\begin{frame}
  \frametitle{3D Linear Approximations}
  \begin{itemize}
  \item moving to two independent variables has the form
    \begin{itemize}
    \item $f(x_0 + \Delta x, y_0 + \Delta y) \approx f(x_0, y_0) + \frac{\partial f}{\partial x}(x_0, y_0) \Delta x + \frac{\partial f}{\partial x}(x_0, y_0) \Delta y$
    \end{itemize}
    \item Can simply add extra terms for each of the additional coordinates
  \end{itemize}
\end{frame}
  
\end{document}

