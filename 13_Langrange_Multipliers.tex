\documentclass{beamer}
\usetheme{Boadilla}

\title{Lagrange Multipliers}

\author{Joey Bernard}
\institute{University of New Brunswick}
\date{\today}

\begin{document}

\begin{frame}
  \titlepage
\end{frame}

\begin{frame}
  \frametitle{Constrained Optimization Problem}
  \begin{itemize}
  \item We have a problem of the form
    \begin{itemize}
    \item Find the maximum and minimum values of the function f(x,y) for (x,y) on the curve g(x,y) = 0
    \end{itemize}
  \item where
    \begin{itemize}
    \item f(x,y) is the objective function
    \item g(x,y) is the constraint function
    \end{itemize}
  \end{itemize}
\end{frame}

\begin{frame}
  \frametitle{Lagrange Multipliers}
  \begin{itemize}
  \item Let f(x,y,z) and g(x,y,z) have continuous first partial derivatives in a region of $R^3$ that contains the surface S given by the equation g(x,y,z) = 0. Further assume that $\nabla g(x,y,z) \ne 0$ on S.
  \item If f, restricted to the surface S, has a local extreme value at the point (a,b,c) on s, then there is a real number $\lambda$ such that
    \begin{itemize}
    \item $\nabla f(a,b,c) = \lambda \nabla g(a,b,c)$
    \end{itemize}
  \item that is
    \begin{itemize}
    \item $f_x(a,b,c) = \lambda g_x(a,b,c)$
    \item $f_y(a,b,c) = \lambda g_y(a,b,c)$
    \item $f_z(a,b,c) = \lambda g_z(a,b,c)$
    \end{itemize}
  \item The number $\lambda$ is called a Lagrange Multiplier
  \end{itemize}
\end{frame}

\end{document}

