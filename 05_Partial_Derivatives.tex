\documentclass{beamer}
\usetheme{Boadilla}

\title{Partial Derivatives}
\author{Joey Bernard}
\institute{University of New Brunswick}
\date{\today}

\begin{document}

\begin{frame}
  \titlepage
\end{frame}

\begin{frame}
  \frametitle{Definitions}
  Remember the limit definition of a derivative
  \begin{itemize}
  \item $ f'(a) = \lim_{h\to\infty} \frac{f(a+h) - f(a)}{h}$
  \end{itemize}

  For notation, we have
  \begin{itemize}
  \item $\frac{df}{dx}$  -  full derivative with respect to x
  \item $\frac{\partial f}{\partial x}$  -  partial derivative with respect to x, also written as $f_x$
  \item These are almost never the same
  \end{itemize}

  So, the limit definition of a partial derivative is given by
  \begin{itemize}
  \item $f_x(x, y) = \lim_{h\to\infty} \frac{f(x+h, y) - f(x, y)}{h}$
  \item can also be written equivalently for $f_y$
  \end{itemize}
  
\end{frame}

\end{document}

